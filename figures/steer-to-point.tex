\documentclass[margin=5mm]{standalone}
\usepackage[T1]{fontenc}
\usepackage[utf8]{inputenc}
\usepackage{pgf,tikz, pgfplots}
\usetikzlibrary{calc}
\usetikzlibrary{intersections}
\pgfplotsset{compat=newest}

\makeatletter
\newcommand{\gettikzxy}[3]{%
  \tikz@scan@one@point\pgfutil@firstofone#1\relax
  \edef#2{\the\pgf@x}%
  \edef#3{\the\pgf@y}%
}
\makeatother

  \begin{document}

  \begin{tikzpicture}[node distance=20mm, anchor=north,
    waypoint/.style={circle, inner sep=0.4mm, red!60!black, draw},
    robotpath/.style={dashed, thick, blue!30!white},
    lookcircle/.style={dotted, green!60!black},
    ]

    \node[coordinate] (O) at (0,0) {};
    \def\pxx{4}
    \def\pyy{3}
    \pgfmathsetmacro{\RR}{(\pxx*\pxx + \pyy*\pyy)/2/\pyy}
    \pgfmathsetmacro{\aa}{\RR - \pyy}

    \node[waypoint, anchor=center, pin=60:{$p_g$}] (wp) at (-\pyy, \pxx) {};
    \node[coordinate] (icr) at (-\RR, 0) {};
    
    \draw[->] (0,0) to node[below, pos=0.99] {$y_b$} (-6cm, 0) ;
    \draw[->] (0,0) to (0, 6cm) node[right] {$x_b$};

    \draw[robotpath] (0,0) arc[radius = \RR, start angle = 0, end angle = 80];

    \draw[] (-\RR, 0) -- ++(0, -2mm) node[below] {$R$};
    \draw[] (-\pyy, 0) -- ++(0, -2mm) node[below] {$p_y$};
    \draw[] (0, \pxx) -- ++(2mm, 0) node[right] {$p_x$};

    \draw[green!70!black] (wp) -- node[left] {$R$} (icr);
    \draw[green!70!black] (wp) -- node[right] {$p_x$} (-\pyy, 0);
    \draw[green!70!black]  (-\pyy, 0) -- node[above] {$a$} (icr);


  \end{tikzpicture}
\end{document}
