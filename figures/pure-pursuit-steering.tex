\documentclass[margin=5mm]{standalone}
\usepackage[T1]{fontenc}
\usepackage[utf8]{inputenc}
\usepackage{pgf,tikz, pgfplots}
\usetikzlibrary{calc}
\usetikzlibrary{intersections}
\pgfplotsset{compat=newest}

\makeatletter
\newcommand{\gettikzxy}[3]{%
  \tikz@scan@one@point\pgfutil@firstofone#1\relax
  \edef#2{\the\pgf@x}%
  \edef#3{\the\pgf@y}%
}
\makeatother

  \begin{document}

  \begin{tikzpicture}[node distance=20mm, anchor=north,
    waypoint/.style={circle, inner sep=0.4mm, red!60!black, draw},
    robotpath/.style={dashed, blue!30!white},
    lookcircle/.style={dotted, green!60!black},
    ]

    \node[coordinate] (O) at (0,0) {};
    \def\Lone{5}
    \pgfmathsetmacro{\cone}{\Lone*cos(255)}
    \pgfmathsetmacro{\sone}{\Lone*sin(255)}

    \draw[->] (0,0) to node[below, pos=0.99] {$y_b$} (-6cm, 0) ;
    \draw[->] (0,0) to (0, 6cm) node[right] {$x_b$};

  \node[waypoint, pin=180:{$p_k$}] (wp1) at (-2.2cm,-3cm) {};
  \node[waypoint, pin=180:{$p_{k+1}$}] (wp2) at (-3cm, 5cm) {};
  \node[waypoint, pin=180:{$p_{k+2}$}] (wp3) at (1cm,8cm) {};

  \draw[name path=pth, robotpath] (wp1) to (wp2);
  \draw[robotpath] (wp2) to (wp3);
  \draw[robotpath] (wp3) -- ++(2, 0.5cm);

  \draw[name path=circ, lookcircle] (0, -\Lone) arc[radius=\Lone, start angle=270, end angle=45];
  \draw[<->, green!60!black] (0,0) -- node[left] {$L$} (\cone , \sone);
  \path [name intersections={of=pth and circ,by=E}];
  \node[waypoint, pin=180:{$p_g$}] at (E) {};
  \gettikzxy{(E)}{\ax}{\ay} 
  \pgfmathsetmacro{\axcm}{\ax*1pt/1cm}
  \pgfmathsetmacro{\RR}{\Lone*\Lone/2/\axcm}

  \draw[dotted, black!80] (E) -- (E -| 0,0) node[right, black] {\large $x$};
  \draw[dotted, black!80] (E) -- (0,0 -| E) node[below, black] {\large $y$};

  
  \draw[dashed, red!60!black] (\RR cm, 0) -- node[below, pos=0] {$R$} node[left] {$R$} (E);
  
  \draw[dotted, orange!80!white] (0,0) arc[radius=-\RR, start angle=0, end angle=80];
  
\end{tikzpicture}
\end{document}
